\documentclass[american,]{article}
\usepackage{lmodern}
\usepackage{amssymb,amsmath}
\usepackage{ifxetex,ifluatex}
\usepackage{fixltx2e} % provides \textsubscript
\ifnum 0\ifxetex 1\fi\ifluatex 1\fi=0 % if pdftex
  \usepackage[T1]{fontenc}
  \usepackage[utf8]{inputenc}
\else % if luatex or xelatex
  \ifxetex
    \usepackage{mathspec}
  \else
    \usepackage{fontspec}
  \fi
  \defaultfontfeatures{Ligatures=TeX,Scale=MatchLowercase}
\fi
% use upquote if available, for straight quotes in verbatim environments
\IfFileExists{upquote.sty}{\usepackage{upquote}}{}
% use microtype if available
\IfFileExists{microtype.sty}{%
\usepackage{microtype}
\UseMicrotypeSet[protrusion]{basicmath} % disable protrusion for tt fonts
}{}
\usepackage[margin=1in]{geometry}
\usepackage{hyperref}
\hypersetup{unicode=true,
            pdftitle={Title},
            pdfauthor={Stuart Miller, Paul Adams, and Chance Robinson},
            pdfborder={0 0 0},
            breaklinks=true}
\urlstyle{same}  % don't use monospace font for urls
\ifnum 0\ifxetex 1\fi\ifluatex 1\fi=0 % if pdftex
  \usepackage[shorthands=off,main=american]{babel}
\else
  \usepackage{polyglossia}
  \setmainlanguage[variant=american]{english}
\fi
\usepackage{natbib}
\bibliographystyle{apalike}
\usepackage{longtable,booktabs}
\usepackage{graphicx,grffile}
\makeatletter
\def\maxwidth{\ifdim\Gin@nat@width>\linewidth\linewidth\else\Gin@nat@width\fi}
\def\maxheight{\ifdim\Gin@nat@height>\textheight\textheight\else\Gin@nat@height\fi}
\makeatother
% Scale images if necessary, so that they will not overflow the page
% margins by default, and it is still possible to overwrite the defaults
% using explicit options in \includegraphics[width, height, ...]{}
\setkeys{Gin}{width=\maxwidth,height=\maxheight,keepaspectratio}
\IfFileExists{parskip.sty}{%
\usepackage{parskip}
}{% else
\setlength{\parindent}{0pt}
\setlength{\parskip}{6pt plus 2pt minus 1pt}
}
\setlength{\emergencystretch}{3em}  % prevent overfull lines
\providecommand{\tightlist}{%
  \setlength{\itemsep}{0pt}\setlength{\parskip}{0pt}}
\setcounter{secnumdepth}{5}
% Redefines (sub)paragraphs to behave more like sections
\ifx\paragraph\undefined\else
\let\oldparagraph\paragraph
\renewcommand{\paragraph}[1]{\oldparagraph{#1}\mbox{}}
\fi
\ifx\subparagraph\undefined\else
\let\oldsubparagraph\subparagraph
\renewcommand{\subparagraph}[1]{\oldsubparagraph{#1}\mbox{}}
\fi

%%% Use protect on footnotes to avoid problems with footnotes in titles
\let\rmarkdownfootnote\footnote%
\def\footnote{\protect\rmarkdownfootnote}

%%% Change title format to be more compact
\usepackage{titling}

% Create subtitle command for use in maketitle
\newcommand{\subtitle}[1]{
  \posttitle{
    \begin{center}\large#1\end{center}
    }
}

\setlength{\droptitle}{-2em}

  \title{Title}
    \pretitle{\vspace{\droptitle}\centering\huge}
  \posttitle{\par}
    \author{Stuart Miller, Paul Adams, and Chance Robinson}
    \preauthor{\centering\large\emph}
  \postauthor{\par}
      \predate{\centering\large\emph}
  \postdate{\par}
    \date{Master of Science in Data Science, Southern Methodist University, USA}

\usepackage{amsmath}
\usepackage[utf8]{inputenc}
\usepackage[T1]{fontenc}
\usepackage{setspace}
\onehalfspacing
\setcitestyle{round}
\newcommand\numberthis{\addtocounter{equation}{1}\tag{\theequation}}

\begin{document}
\maketitle
\begin{abstract}
Abstract
\end{abstract}

\hypertarget{introduction}{%
\section{Introduction}\label{introduction}}

\citet{Sleuth}

\hypertarget{ames-iowa-data}{%
\section{Ames, Iowa Data}\label{ames-iowa-data}}

\citet{Kaggle2016}

\hypertarget{analysis-question-i}{%
\section{Analysis Question I}\label{analysis-question-i}}

\hypertarget{question-of-interest}{%
\subsection{Question of Interest}\label{question-of-interest}}

Restatement of the problem

\hypertarget{modeling}{%
\subsection{Modeling}\label{modeling}}

TODO: Build and fit the model

We will consider two models: (1) the logarithm of sale price as the
response of living room area and (2) the logarithm of sale price as the
response of living room area accounting for differences in the three
neighborhood of interest (Brookside, Northwest Ames, and Edwards) where
Edwards will be used as the reference.

\textbf{Reduced Model}

\begin{equation}
\mu \lbrace log(SalePrice) \rbrace = \beta_0 + \beta_1(LivingRoomArea) \label{eq:reduced}
\end{equation}

\textbf{Full Model}

\begin{align}
\mu \lbrace log(SalePrice) \rbrace = \beta_0 + \beta_1(LivingRoomArea) +  \beta_2(Brookside) +\beta_3(NorthwestAmes) + \nonumber\\
\beta_3(Brookside)(LivingRoomArea) + \beta_4(NorthwestAmes)(LivingRoomArea) \label{eq:full}
\end{align}

We will use an extra sums of square test to verify that the interaction
terms are useful for the model. The ESS test provides convincing
evidence that the interaction terms are useful for the model (p-value
\textless{} 0.0001); thus, we will continue with the full model.

\begin{verbatim}
## Analysis of Variance Table
## 
## Model 1: log(SalePrice) ~ (GrLivArea) + Neighborhood_BrkSide + Neighborhood_NAmes
## Model 2: log(SalePrice) ~ (GrLivArea) + Neighborhood_BrkSide + Neighborhood_NAmes + 
##     (GrLivArea) * Neighborhood_BrkSide + (GrLivArea) * Neighborhood_NAmes
##   Res.Df    RSS Df Sum of Sq      F    Pr(>F)    
## 1    377 14.824                                  
## 2    375 13.441  2    1.3834 19.299 1.053e-08 ***
## ---
## Signif. codes:  0 '***' 0.001 '**' 0.01 '*' 0.05 '.' 0.1 ' ' 1
\end{verbatim}

\hypertarget{model-assumption-assessment}{%
\subsection{Model Assumption
Assessment}\label{model-assumption-assessment}}

Address each assumption

\begin{center}\includegraphics{HousePriceRegressionAnalysis_files/figure-latex/diag-plots-1} \end{center}

\hypertarget{comparing-competing-model}{%
\subsection{Comparing Competing Model}\label{comparing-competing-model}}

Adj R\^{}2

\hypertarget{parameters}{%
\subsection{Parameters}\label{parameters}}

\begin{itemize}
\tightlist
\item
  Estimates
\item
  Influential points
\item
  Residual plots
\end{itemize}

\hypertarget{conclusion}{%
\subsection{Conclusion}\label{conclusion}}

A short summary of teh analysis

\hypertarget{analysis-question-ii}{%
\section{Analysis Question II}\label{analysis-question-ii}}

\hypertarget{appendix}{%
\section{Appendix}\label{appendix}}

\renewcommand\refname{References}
\bibliography{references.bib}


\end{document}
